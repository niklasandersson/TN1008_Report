Simulating the behavior of fluids is an intriguing topic that has grown into an
expansive field of research and development. This broad area is mainly divided
into two different categories of application for fluid simulations. The first
category deals with attempting to create a simulation that is as physically
correct as possible while the second is focusing on creating a simulation that
is capable of running in real time but whose behavior might not be entirely
accurate. In this report we focus solely on the second category and discuss a
method called \textit{Position-Based Dynamics} (PBD) \cite{muller2007position}
that is designed to be used in games and other real-time applications.
Position-based dynamics is a particle-based method which means that the
behavior of the fluid is determined by the movements of a large number of
particles. This is not something that is unique to Position-Based Dynamics by
any means as it is in fact one of the most common ways of doing fluid
simulation. Another common method that is based on particles is
\textit{Smoothed-Particle Hydrodynamics} (SPH) \cite{monaghan1992smoothed}.
What makes PBD unique is that it is based on a set of constraints directly
affecting the positions of the particles while merely glancing over the
traditional concepts of acceleration, velocities and forces.
