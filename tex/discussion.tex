Our implementation of a fluid simulation using position-based dynamics was able
to achieve real-time performance for a large number of particles even while
running on laptop grade hardware. By varying parameters such as vorticity and
viscosity different fluid behaviors can be achieved ranging form thick slow
moving substances to quick and turbulent water.

In order for the simulation to be running in real time it was necessary to
exploit high levels of parallelism. Initial tests where conducted in the form
of single threaded applications running on the CPU but we where only able to
simulate less than 500 particles before the framerate became unacceptably low.
Even running the simulation in a multithreaded manner on modern CPUs capable of
8 or more concurrent tasks would not be sufficient to achieve the desired
performance. The solution to this problem was to utilize the massively parallel
architecture of current GPUs (Graphics Processing Unit) that are often times
capable of over 1000 simultaneously executing tasks. The downside to using GPUs
is that due to their unique architecture they make creating programs for them
more difficult than something running on a CPU.

\subsection{Future Work}
Position-based dynamics is not limited to simulating
fluids and can be used to model a large variety of behaviors. The possibilities
of position-based dynamics are only limited to the number of constraints that
can be expressed. It would be of interest to further develop our software to
handle materials such as cloth as described in~\cite{muller2007position} using
stretching constraints or deformation of objects~\cite{muller2005meshless} by
incorporating shape matching constraints.

It would also be of interest to further extend the fluid simulation as
described in~\cite{macklin2014unified} to allow for fluids of varying densities
to interact. This would include changing how density is estimated between
neighboring particles to accommodate for a per particle variation in density
contribution. The result of this would be that phenomena such as oil floating
on top of water could be simulated.
