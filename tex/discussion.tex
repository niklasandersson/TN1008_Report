From the beginning our goal was to create a GPU framework like
\cite{macklin2014unified} supporting simulation of many different elements,
e.g. fluids, cloth and rigid bodies. Therefore we started of studying how
Position-Based Dynamics worked in general. When we had an understanding of what
we would need to do in order to achieve our goal we decided to start
implementing prototype applications solving the constraints we were interested
in. The selected constraints were shape matching constraints for rigid and
deformable bodies and density constraints for fluids.

These prototypes had simulations running in 2D and were made single threaded to
allow more convenient debugging on the CPU. However as the project progressed
and when we were about to make the switch to the GPU we decided to narrow our
scope. Hence our focus shifted into solely develop a fluid simulation running
on the GPU in 3D.

We first had a stint working with OpenCL and OpenGL interoperability. But after
a longer break we decided to make the switch to CUDA and OpenGL
interoperability instead. This as we were keen on using CUDA \textit{Surfaces}
which are textures that can be both written and read during the same kernel
run.

In the remainder of this section we will in greater detail explain the
implementations of the prototypes and the final fluid simulation.
