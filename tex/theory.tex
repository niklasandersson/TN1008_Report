In this secion we present the core concepts of PBD and
explain the theory involed in creating a fluid simulation.

\subsection{Position-Based Dynamics}
In position-based dynamics particles constitutes the fundamental component of
any simulation.  Particles are represented using three attributes. The first
attribute is the position of the particle, the second is the particles velocity
and the third is its mass. By restricting the basis of any simulation to a
simple particle representation several different phenomenon, such as water,
smoke, cloth and deformable bodies, can be simulated in a unified manner by
varying the behavior of the particles.

In order to control the movements of the particles and obtain a simulation with
the desired behavior a set of constraints are used. A constraint can be defined
in one of two ways. The first is as a bilateral constrain as defined in
Equation~\ref{eq:bilateralC} and the second is a a unilateral constraint as in
Equation~\ref{eq:unilateralC}. In these equations $ \mathbf{x_{i}} $ denotes
the position of particle $ i $.

\begin{equation}
\label{eq:bilateralC}
C(\mathbf{x_{1}}, \mathbf{x_{2}}, ..., \mathbf{x_{n}}) = 0
\end{equation}

\begin{equation}
\label{eq:unilateralC}
C(\mathbf{x_{1}}, \mathbf{x_{2}}, ..., \mathbf{x_{n}}) \geq 0
\end{equation}

In order to ensure that all constraints are fullfilled $ \Delta \mathbf{x} $ is
introduced as seen in Equations~\ref{eq:bilateralCDelta} and~\ref{eq:unilateralCDelta}.

\begin{equation}
\label{eq:bilateralCDelta}
C(\mathbf{x_{1}} + \Delta \mathbf{x_{1}}, \mathbf{x_{2}} + \Delta \mathbf{x_{2}}, ..., \mathbf{x_{n}} + \Delta \mathbf{x_{n}}) = 0
\end{equation}

\begin{equation}
\label{eq:unilateralCDelta}
C(\mathbf{x_{1}} + \Delta \mathbf{x_{1}}, \mathbf{x_{2}} + \Delta \mathbf{x_{2}}, ..., \mathbf{x_{n}} + \Delta \mathbf{x_{n}}) \geq 0
\end{equation}

For a more compact notation Equations~\ref{eq:bilateralCDelta}
and~\ref{eq:unilateralCDelta} can be written as Equation~\ref{eq:cCombined}
where $ \succ $ is either $ = $ or $ \geq $.

\begin{equation}
\label{eq:cCombined}
C(\mathbf{x} + \Delta \mathbf{x}) \succ 0
\end{equation}

As described in~\cite{macklin2013position} Equation~\ref{eq:Taylor} can be
derived from Equation~\ref{eq:cCombined} using Taylor expansion.

\begin{equation}
\label{eq:Taylor}
C(\mathbf{x} + \Delta \mathbf{x}) \approx C(\mathbf{x}) + \nabla C(\mathbf{x})^{T} \Delta \mathbf{x} \succ 0
\end{equation}

To further extend Equation~\ref{eq:Taylor} it is assumed that $ \Delta
\mathbf{x} $ is restricted to be in the direction of the gradient as expresed
in Equation~\ref{eq:deltaX}.

\begin{equation}
\label{eq:deltaX}
\Delta \mathbf{x} = \nabla C(\mathbf{x}) \lambda
\end{equation}

Inserting Equation~\ref{eq:deltaX} into Equation~\ref{eq:Taylor} and solving for $ \lambda $ results in Equation~\ref{eq:solvedforlambda}.

\begin{equation}
\label{eq:solvedforlambda}
\lambda = - \frac{C(\mathbf{x})}{\left | \nabla C(\mathbf{x}) \right |^2}
\end{equation}

Equation~\ref{eq:solvedforlambda} can be extended to
Equation~\ref{eq:lambdaallparticles} when considering all particles involed in
the constraint.

\begin{equation}
\label{eq:lambdaallparticles}
\lambda = - \frac{C(\mathbf{x_{1}}, \mathbf{x_{2}}, ..., \mathbf{x_{n}})}{\sum_{j} \left | \nabla_{x_{j}} C(\mathbf{x_{1}, \mathbf{x_{2}}, ..., \mathbf{x_{n}})} \right |^2}
\end{equation}

By calculating $ \Delta \mathbf{x} $ during each time step of the simulation
and applying this to the postion of each particle according to
Equation~\ref{eq:applydelta} the behavior of the particles will accommodate the
constraints placed on them.

\begin{equation}
\label{eq:applydelta}
\mathbf{x_{i}^{'}} = \mathbf{x_{i}} + \Delta \mathbf{x_{i}} = \mathbf{x_{i}} + \lambda \nabla_{x_{i}} C(\mathbf{x_{1}}, \mathbf{x_{2}}, ..., \mathbf{x_{n}})
\end{equation}




\subsection{Collision constraint}
The collision constraint between two particles $ \mathbf{x}_{i} $ and $
\mathbf{x}_{j} $ is formulated below, where
$ r $ is the radius of the particles and $ \mathbf{x_{ij}} = \mathbf{x_{i}} - \mathbf{x_{j}} $.

\begin{equation} \label{eq:collisionConstraint}
  C(\mathbf{x}_{i}, \mathbf{x}_{j}) = \| \mathbf{x}_{ij} \| - r \geq 0
\end{equation}

The gradient of the collision constraint with respect to each particle is the
normalized collision vector as defined as follows:

\begin{equation}
\label{eq:collisiongradient}
\begin{aligned}
\nabla_{\mathbf{x_{i}}} C(\mathbf{x_{i}}, \mathbf{x_{j}}) = \mathbf{n} = \frac{\mathbf{x_{ij}}}{\left | \mathbf{x_{ij}} \right |}
\\
\nabla_{\mathbf{x_{j}}} C(\mathbf{x_{i}}, \mathbf{x_{j}}) = - \mathbf{n} = - \frac{\mathbf{x_{ij}}}{\left | \mathbf{x_{ij}} \right |}
\end{aligned}
\end{equation}

By substituting Equation~\ref{eq:collisionConstraint} and
Equation~\ref{eq:collisiongradient} into Equation~\ref{eq:lambdaallparticles}
we can derive an expression for $ \lambda $ for both particles in the
constraint:

\begin{equation}
\label{eq:lambdacollision}
\lambda_{i} = \lambda_{j} = \frac{\left | \mathbf{x_{ij}} \right | - r}{w_{i} + w_{j}}
\end{equation}

In order to calculate the two correction vectors, $ \Delta \mathbf{x_{i}} $ and
$ \Delta \mathbf{x_{j}} $, for the position of each particle we insert
Equation~\ref{eq:lambdacollision} and Equation~\ref{eq:collisiongradient} into
Equation~\ref{eq:deltaX} and arrive at the following result:

\begin{equation}
\label{eq:collisionresult}
\begin{aligned}
\Delta \mathbf{x_{i}} = \frac{w_{i}}{w_{i} + w_{j}}(\left | \mathbf{x_{ij}} \right | - r) \frac{\mathbf{x_{ij}}}{\left | \mathbf{x_{ij}} \right |}
\\
\Delta \mathbf{x_{j}} = -\frac{w_{j}}{w_{i} + w_{j}}(\left | \mathbf{x_{ij}} \right | - r) \frac{\mathbf{x_{ij}}}{\left | \mathbf{x_{ij}} \right |}
\end{aligned}
\end{equation}



\subsection{Density constraint}

The estimation of the predicted position of the particles is initially based on
solving a density constraint $C$ at Equation $\ref{eq:Taylor}$. Where each
density constraint is applied per particle. The density constraint is a
function of the position of the current particle and the positions of its
neighbours within a fixed radius. For each particle \textit{i}, the applied
density constraint is defined as Equation \ref{eq:Ci}

\begin{equation}
\label{eq:Ci}
C_i(\mathbf{x}) = \frac{\rho_i}{\rho_0} - 1,
\end{equation}

where $\mathbf{x}$ contains the positions of the neighbouring particles,
$\rho_0$ is the fluid's rest density and $\rho_i$ is estimated by the density
SPH estimator:

\begin{equation}
\label{eq:Rhoi}
\rho_i = \sum\limits_{j} w_j W(\mathbf{x}_i - \mathbf{x}_j, h).
\end{equation}

In this particular case, where all particles have equal mass, the mass
parameter $m_j$ can be removed from Equation $\ref{eq:Rhoi}$. What is left is a
sum of density kernels $W$, also called \textit{Poly6}, see
\cite{muller2003particle}. Where $W$ is a kernel function with the parameters;
the position of the current particle  $i$, neighboring positions $j$ and a
fixed radius $h$.

The next term to be solved from Equation $\ref{eq:Taylor}$ is $\nabla C$. It is
done by following the approach of \cite{macklin2013position} and has two different outcomes based on whenever particle $k$ is a neighbouring particle or not:

\begin{equation}
 \nabla \mathbf{x}_k C_i = \frac{1}{\rho_0}
  \begin{cases}
  \label{eq:NablaC}
   \displaystyle \sum\limits_{j} \nabla \mathbf{x}_k W(\mathbf{x}_i - \mathbf{x}_j, h) & $\text{if }$ k = i \\
   - \nabla \mathbf{x}_k W(\mathbf{x}_i - \mathbf{x}_j, h) & $\text{if }$ k = j \\
  \end{cases}
\end{equation}

The definition of $ \nabla \mathbf{x} W(\mathbf{x}_i - \mathbf{x}_j, h) $ is a
kernel called \textit{Spiky} that is, together with the \textit{Poly6}
kernel, defined and described in ~\cite{muller2003particle}.

As seen in $\cite{macklin2013position}$, we can observe that this is not stable
when the particles are at the boundary of the smoothing kernel, mainly because
of the denominator with the exponents in the kernel function.  Therefore a user
specified relaxation parameter, $\varepsilon$, is added to Equation $\ref{eq:Taylor}$ to add constraint force. The modified version of Equation $\ref{eq:lambdaallparticles}$ is given by

\begin{equation}
\label{eq:LambdaEpsilon}
\lambda_i = \frac{- C_i(\mathbf{x}) }{ \sum\limits_{k} |\nabla \mathbf{x}_k C_i|^2 + \varepsilon}.
\end{equation}
We can now include $\lambda_j$ from the neighbouring particles in the
estimation of the final position update, $\Delta \mathbf{x}$, given by

\begin{equation}
\label{eq:DeltaP}
\Delta \mathbf{x}_i = \frac{1}{\rho_0} \sum\limits_{j} (\lambda_i + \lambda_j) \nabla W(\mathbf{x}_i - \mathbf{x}_j, h).
\end{equation}

\subsection{Tensile instability} We continue to follow the theory in
\cite{macklin2013position} that describes different problems with particle clustering and coupling as a result of negative pressure. The trick is to force the pressure to be non-negative using an additional pressure term. This pressure
force is described as

\begin{equation}
\label{eq:Scorr}
s_{corr} = -k \left( \frac{W(\mathbf{x}_i - \mathbf{x}_j, h)}{W(\Delta \mathbf{q}, h)} \right),
\end{equation}

where $\Delta \mathbf{q}$ is a point with a fixed distance from the current particle inside the kernel
radius and k is a small positive constant. This term is then included in
Equation $\ref{eq:DeltaP}$ and the final version is updated to

\begin{equation}
\label{eq:DeltaPscorr}
\Delta \mathbf{x} = \frac{1}{\rho_0} \sum\limits_{j} (\lambda_i + \lambda_j + s_{corr}) \nabla W(\mathbf{x}_i - \mathbf{x}_j, h).
\end{equation}

\subsection{Vorticity confinement and viscosity} 
In order to add turbulent motion to the fluid we add a \textit{vorticity} term to the calculations, as described in
\cite{macklin2013position}. The vorticity is calculated by first using the
Spiky kernel function as following:

\begin{equation}
\label{eq:Omega}
\omega_{i} = \nabla \times \mathbf{v} =  \sum\limits_{j} \mathbf{v}_{ij} \times \nabla_{\mathbf{x}_{j}} W(\mathbf{x}_{i} - \mathbf{x}_{j}, h),
\end{equation}

where $\mathbf{v}_{ij} = \mathbf{v}_{j} - \mathbf{v}_{i}$, and then by using the
vector $\mathbf{N} = \frac{\eta}{|\eta|}$, with $\eta = \nabla|\omega_{i}|$,
which will have a direction from areas with low vorticity towards areas
containing high vorticity. The resulting vorticity force is defined by

\begin{equation}
\label{eq:Vorticity}
\mathbf{f}_{i_{vorticity}} = \varepsilon \left(\mathbf{N} \times \omega_{i} \right).
\end{equation}

The \textit{viscosity} is added to the velocity of the particles', the term
will control the fluid's consistency and is calculated as

\begin{equation}
\label{eq:Viscosity}
\mathbf{v}_{i}^{new} = \mathbf{v}_{i} + c \sum\limits_{j} \mathbf{v}_{ij} \cdot W(\mathbf{x}_i - \mathbf{x}_j, h).
\end{equation}

