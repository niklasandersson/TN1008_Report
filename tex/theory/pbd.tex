In position-based dynamics particles constitutes the fundamental component of
any simulation.  Particles are represented using three attributes. The first
attribute is the position of the particle, the second is the particle velocity
and the third is its mass. By restricting the basis of any simulation to a
simple particle representation several different phenomenon, such as water,
smoke, cloth and deformable bodies, can be simulated in a unified manner by
varying the behavior of the particles.

In order to control the movements of the particles and obtain a simulation with
the desired behavior a set of constraints are used. A constraint can be defined
in one of two ways. The first is as a bilateral constraint as defined in
Equation~\ref{eq:bilateralC} and the second is a a unilateral constraint as in
Equation~\ref{eq:unilateralC}. In these equations $ \mathbf{x_{i}} $ denotes
the position of particle $ i $.

\begin{equation}
\label{eq:bilateralC}
C(\mathbf{x_{1}}, \mathbf{x_{2}}, ..., \mathbf{x_{n}}) = 0
\end{equation}

\begin{equation}
\label{eq:unilateralC}
C(\mathbf{x_{1}}, \mathbf{x_{2}}, ..., \mathbf{x_{n}}) \geq 0
\end{equation}

In order to ensure that all constraints are fullfilled $ \Delta \mathbf{x} $ is
introduced as seen in Equations~\ref{eq:bilateralCDelta} and~\ref{eq:unilateralCDelta}.

\begin{equation}
\label{eq:bilateralCDelta}
C(\mathbf{x_{1}} + \Delta \mathbf{x_{1}}, \mathbf{x_{2}} + \Delta \mathbf{x_{2}}, ..., \mathbf{x_{n}} + \Delta \mathbf{x_{n}}) = 0
\end{equation}

\begin{equation}
\label{eq:unilateralCDelta}
C(\mathbf{x_{1}} + \Delta \mathbf{x_{1}}, \mathbf{x_{2}} + \Delta \mathbf{x_{2}}, ..., \mathbf{x_{n}} + \Delta \mathbf{x_{n}}) \geq 0
\end{equation}

For a more compact notation Equations~\ref{eq:bilateralCDelta}
and~\ref{eq:unilateralCDelta} can be written as Equation~\ref{eq:cCombined}
where $ \succ $ is either $ = $ or $ \geq $.

\begin{equation}
\label{eq:cCombined}
C(\mathbf{x} + \Delta \mathbf{x}) \succ 0
\end{equation}

As described in~\cite{macklin2013position} Equation~\ref{eq:Taylor} can be
derived from Equation~\ref{eq:cCombined} using \textit{Taylor expansion}.

\begin{equation}
\label{eq:Taylor}
C(\mathbf{x} + \Delta \mathbf{x}) \approx C(\mathbf{x}) + \nabla C(\mathbf{x})^{T} \Delta \mathbf{x} \succ 0
\end{equation}

To further extend Equation~\ref{eq:Taylor} it is assumed that $ \Delta
\mathbf{x} $ is restricted to be in the direction of the gradient as expresed
in Equation~\ref{eq:deltaX}.

\begin{equation}
\label{eq:deltaX}
\Delta \mathbf{x} = \nabla C(\mathbf{x}) \lambda
\end{equation}

Inserting Equation~\ref{eq:deltaX} into Equation~\ref{eq:Taylor} and solving for $ \lambda $ results in Equation~\ref{eq:solvedforlambda}.

\begin{equation}
\label{eq:solvedforlambda}
\lambda = - \frac{C(\mathbf{x})}{\left | \nabla C(\mathbf{x}) \right |^2}
\end{equation}

At this point it is possible to solve for the correction vector $ \Delta
\mathbf{x} $ but to make it more practical we consider
Equation~\ref{eq:solvedforlambda} for each individual particle involved in the
constraint. This will also allow for the inclusion of particles with different
mass. Equation~\ref{eq:lambdaallparticles} shows the result of these
considerations where $ w_{j} $ is the inverse mass of particle $ j $.

\begin{equation}
\label{eq:lambdaallparticles}
\lambda = - \frac{C(\mathbf{x_{1}}, \mathbf{x_{2}}, ..., \mathbf{x_{n}})}{\sum_{j} w_{j} \left | \nabla_{x_{j}} C(\mathbf{x_{1}, \mathbf{x_{2}}, ..., \mathbf{x_{n}})} \right |^2}
\end{equation}

By calculating $ \Delta \mathbf{x} $ during each time step of the simulation
and applying this to the postion of each particle according to
Equation~\ref{eq:applydelta} the behavior of the particles will accommodate the
constraints placed on them. Note that the inverse mass of the particle under
consideration is present here as well.

\begin{equation}
\label{eq:applydelta}
\mathbf{x_{i}^{'}} = \mathbf{x_{i}} + \Delta \mathbf{x_{i}} = \mathbf{x_{i}} + \lambda w_{i} \nabla_{x_{i}} C(\mathbf{x_{1}}, \mathbf{x_{2}}, ..., \mathbf{x_{n}})
\end{equation}

