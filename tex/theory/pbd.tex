In position-based dynamics particles constitutes the fundamental component of
any simulation.  Particles are represented using three attributes. The first
attribute is the position of the particle, the second is the particles velocity
and the third is its mass. By restricting the basis of any simulation to a
simple particle representation several different phenomenon, such as water,
smoke, cloth and deformable bodies, can be simulated in a unified manner by
varying the behavior of the particles.

In order to control the movements of the particles and obtain a simulation with
the desired behavior a set of constraints are used. A constraint can be defined
in one of two ways. The first is as a bilateral constrain as defined in
Equation~\ref{eq:bilateralC} and the second is a a unilateral constraint as in
Equation~\ref{eq:unilateralC}. In these equations $ \mathbf{x_{i}} $ denotes
the position of particle $ i $.

\begin{equation}
\label{eq:bilateralC}
  C(\mathbf{x_1}, \mathbf{x_2}, ..., \mathbf{x_n}) = 0
\end{equation}

\begin{equation}
\label{eq:unilateralC}
  C(\mathbf{x_1}, \mathbf{x_2}, ..., \mathbf{x_n}) \geq 0
\end{equation}

In order to ensure that all constraints are fullfilled $ \Delta \mathbf{x} $ is
introduced as seen in Equations

\begin{equation}
\label{eq:bilateralCDelta}
C(\mathbf{x_1} + \Delta \mathbf{x_1}, \mathbf{x_2} + \Delta \mathbf{x_2}, ..., \mathbf{x_n} + \Delta \mathbf{x_}) = 0
\end{equation}

