In order to create a fluid simulation using PBD the two
types of constraints described in the theory section were used. Collision
constraints were used in order to handle collision between pairs of particles
and density constraints were used to give the particles a fluid like behavior.
In this section we will describe how this simulation was implemented.
Algorithm~\ref{alg:overview} presents an outline of the simulation and the steps that are repeated for every time step. This overview will
act as the basis for the remainder of this section.

\begin{algorithm}
\caption{Outline of a simulation step}
\label{alg:overview}
\begin{algorithmic}[1]
\small

\For{$i$ : $numberOfParticles$}
\State Apply forces: $\mathbf{v}_{i} \Leftarrow \mathbf{v}_{i} + (\mathbf{f}_{gravity} + \mathbf{f}_{i_{vorticity}})\Delta t$
\State Predict position: $\mathbf{x}_{i}^{*} \Leftarrow \mathbf{x}_{i} + \mathbf{v}_{i} \Delta t$
\State Confine particle to box, adjust $\mathbf{x}_{i}$, $\mathbf{x}_{i}^{*}$ and $\mathbf{v}_{i}$
\EndFor


\For{$i$ : $numberOfParticles$}
\State Compute cell id $h_{i}$ by utilization of Z-order hashing
\EndFor

\State Sort particles in increasing $h$

\State Reorder all textures and buffers according to sorted ordering

\State Compute $cellStarts$ and $cellEndings$ for all cells containing particles


\For{$i$ : $numberOfParticles$}
\State Find neighbouring particles $N_{i}(\mathbf{x}_{i}^{*})$ used by collision and density constraints
\EndFor

\While{$solverIteration$ $<$ $numberOfSoverIterations$}
\For{$i$ : $numberOfParticles$}
\State Compute lambda $\lambda_{i}$
\EndFor
\For{$i$ : $numberOfParticles$}
\State Compute delta position $\Delta \mathbf{x}_{i}^{*}$
\EndFor
\While{$stabilizationIteration$ $<$ $numberOfStabilizationIterations$}
\State Solve collision constraints, update both $\mathbf{x}$ and $\mathbf{x}^{*}$
\EndWhile
\For{$i$ : $numberOfParticles$}
\State Apply delta position: $\mathbf{x}_{i}^{*} \Leftarrow \mathbf{x}_{i}^{*} + \Delta \mathbf{x}_{i}^{*}$
\EndFor
\EndWhile
\For{$i$ : $numberOfParticles$}
\State Update velocity: $\mathbf{v}_{i} \Leftarrow \frac{(\mathbf{x}_{i}^{*} - \mathbf{x}_{i})}{\Delta t}$
\State Update position: $\mathbf{x}_{i} \Leftarrow \mathbf{x}_{i}^{*}$
\EndFor
\For{$i$ : $numberOfParticles$}
\State Compute omegas $\mathbf{\Omega}_{i}$ to be used for computation of vorticity and viscosity
\EndFor
\For{$i$ : $numberOfParticles$}
\State Compute vorticity $\mathbf{f}_{i_{vorticity}}$
\EndFor
\For{$i$ : $numberOfParticles$}
\State Compute and apply viscosity: $\mathbf{v}_{i} \Leftarrow \mathbf{v}_{i} + \mathbf{v}_{viscosity}$
\EndFor

\end{algorithmic}
\end{algorithm}



\subsection{Predict Positions}
As seen in Algorithm ~\ref{alg:overview} the first thing that needs to be done at the beginning of a time step is to update the velocity of all particles by
considering their current velocity and any external forces applied to them as follows:

\begin{equation}
\label{eq:velocity}
\mathbf{v}_{i}^{new} = \mathbf{v}_{i} + (\mathbf{f}_{gravity} + \mathbf{f}_{i_{vorticity}})\Delta t
\end{equation}

We only consider two external forces $ f_{gravity} $ and $ f_{i_{vorticity}} $
that represents the force contributed by gravity and vorticity calculations
respectively. Note that vorticity is calculated at a later stage but is carried
over to the next time step so the $ f_{i_{vorticity}} $ that gets applied here
was calculated at the previous time step.

Once the velocity has been obtained it is used to predict a new particle
position according to the following expression:

\begin{equation}
\label{eq:predict}
\mathbf{x}_{i}^{*}= \mathbf{x}_{i} + \mathbf{v}_{i}^{new} \Delta t
\end{equation}

Having calculated the predicted position it is also necessary to verify that it
is actually a valid position. This is a requirement since the simulation is
limited to a certain volume. This step is performed by comparing the predicted
position against the boundary of the space containing the simulation. If a
particle is located outside the boundary its position is clamped to be inside
the volume.


\subsection{Collision}
The next step in the simulation is to handle collisions between particles.
Solving collision constraints between particles by comparing every pair of
particles to one another is a very time consuming process,
$\mathcal{O}(n^{2})$. As such it is often desired to look at smarter ways of
solving collisions between particles. Due to the fact that particles and other
objects for that matter only interact in small regions it is well suited to
look at methods involving spatial subdivisions.

\subsubsection{Uniform grid}

One such method that also performs well and is easy to parallelize is the use
of a \textit{Uniform Grid} \cite{Green}. A uniform grid implies that the world
is composed of a number of cells in a cubical grid, where each cell can store
particles. If the cell width equals the diameter of a particle (under the
assumption that all particles have the same diameter) collisions only occur
between particles of neighbouring cells. This means that only 27 cells per
particle needs to be considered in a three dimensional space if the particle
collisions are to be solved in a particle based manner. This type of grid is
also convenient when solving fluid constraints that originates from the ideas
behind SPH, as those require all neighbouring particles within a certain
distance (defined by the kernel width of \textit{Spiky} and \textit{Poly6}
kernels).

\subsubsection{Collision with objects}

By transforming the geometry of an object to a particle representation and
adding shape matching constraints to hold the particles together one can also
have collisions with arbitrary objects \cite{muller2005meshless,
macklin2014unified}. The process of creating a particle representation of an
object is called \textit{Voxelization} and there exist various methods for
achieving this, e.g. \cite{VoxPolygon, VoxSingle}. Some methods build
\textit{Occtrees} to be used for querying occupied \textit{Voxels} while other
methods derive the particle representation from \textit{Ray-casting}
\cite{VoxSingle}. Performance wise it is suitable to perform the Voxelization
of objects prior to the start of the simulation.

\subsection{Solving collision constraints}

For solving collision constraints we use the method in
\cite{Green} that are based upon sorting grid cells of a uniform grid. First is
the buffer that is going to store the cell ids filled with the maximum value of
an unsigned integer.  Then, per particle, is a cell id computed and stored in
the cell ids buffer at the location represented by the index of the particle.
The choice of cell id can for example be done by a linear indexing of the cells
or as we do, by using a space filling curve. We use the \textit{Z-order curve}
that is based on calculations of \textit{Morton codes} as it will increase the
spatial locality of nieghbouring particles in the buffer \cite{Green}. Higher
spatial locality between particles is great as the use of textures then will
give more cache hits when we perform calculations per particle involving its
neighbours.

When each particle has received a cell id, the cell ids buffer is sorted
increasingly. The index of each particle is also sorted as part of the sorting
process, i.e. we sort by key, where the cell id is key and the particle index
is value. This, as it will enable reordering of all other buffers and textures
as well when the sorting is done, allowing for more cache hits to happen and
for more convenient reads and writes inside the code.

Now all particles in each grid cell lie next to each other in memory. To then
find neighbouring particles it is convenient to know the start and end of each
cell. These are found by launching kernels per particle, where each particle
compares it's cell id with the cell id of the previous and next particle to see
if they are equal. If the previous particle's cell id is different from this
particle's cell id we know that the starting index of the cell for this
particle is the index of this particle. Vice versa then applies to finding the
ending index of a cell.

Given the cell endings and cell starts the k-nearest neighbours of a particle
can be found. To find the neighbours we first need to find the neighbouring
cells. This is trivial with a linear indexing. But with a space filling curve
we need to first compute a new position and then apply the space filling
curve's hash function to find out the index of that cell. I.e. by adding cell
widths to the position of a particle (in x, y, and z) we can compute the cell
id for that position, the cell id can then be used to find the beginning of
that cell, allowing us to loop through all the particles inside that cell. This
as we know when to stop looping due to us previously deriving the ending index
of the cell.

Once we found the neighbouring particles of each particle there are several
methods that can be applied to solve the collision constraints between them. We
have implemented two different methods. The first method implemented uses
atomics and can either be executed per constraint or per particle. We prefer
the per constraint manner where the following happens.

\begin{enumerate}
\item Each particle $p$ stores it's neighbours in a buffer.
\item Launch kernels for each neighbouring particle $q$ in the buffer.
\item From the index of the element particle $p$ can be derived.
\item We now know that particles $p$ and $q$ are affected by this constraint.
\item Solve the collision constraint and update both $\mathbf{x}_{p}$ and $\mathbf{x}_{p}^{*}$.
\end{enumerate}

Only particle $p$ should be updated as there are duplicates of every collision
constraint, it will namely exist another constraint where $p$ and $q$ are
swapped. Both the actual position ($\mathbf{x}_{p}$) and the predicted position
($\mathbf{x}_{p}^{*}$) are updated as we are only interested in separating the
particles and not to perform a physical collision. To perform read and write
(update) of the positions we must involve atomics as there can be several
threads writing to the same memory address at the same time.

The second method we implemented to solve collision constraints is based on
constructing a number of batches (maximum number of batches needed equals the
number of possible collisions per particle) \cite{bullet}, where each batch
contains a set of collision constraints that can be solved in parallel without
the use of atomics.

Something \cite{bullet} \cite{radix}


\subsection{Solving the Density Constraint}
The workflow for solving the density constraint is closely tied with the theory starting at Equation \ref{eq:Ci}. 
To determine $\lambda_{i}$, the components in Equation \ref{eq:LambdaEpsilon} are calculated separately in parallel 
for each particle \textit{i}.
The numerator is calculated by following Equation \ref{eq:Ci}, which is a sum of the smoothing kernel \textit{Poly6} over 
all neighbours. The smoothing kernel has a width $h_{Poly6} = 4.1$ and the rest density parameter, $\rho$, is set to be $\rho = 1200$. 
The denominator is a sum of the norm of the \textit{Spiky} kernel calculated over all neighbours 
and is then set to the power of 2. The kernel width $h_{Spiky} = 6.1$ for \textit{Spiky} works well and the 
relaxation parameter $\varepsilon$ is set to $\varepsilon = 0.0001$.
\\
\newline
To extend the parallization even more we rearrange and rewrite Equation \ref{eq:LambdaEpsilon} by including Equation \ref{eq:NablaC}. With the 
new Equation \ref{eq:LambdaNew} we can observe that the difference between the sums is the absolute value mark 
outside respectively inside of the sums: 
\\
\begin{equation}
\label{eq:LambdaNew}
\lambda_i = \frac{- C_i(\hat{\mathbf{x}}) }{ |\sum\limits_{k} \nabla \mathbf{x}_k W|^{2} + \sum\limits_{j} |-\nabla \mathbf{x}_j W|^2  + \varepsilon}.
\end{equation}
\\
\newline
We then continue to follow the algorithm overview at Algorithm \ref{alg:overview}, where the next step is to estimate the new position $\Delta \mathbf{x}^{*}_{i}$.
The predicted position is defined at Equation \ref{eq:DeltaPscorr}, containing a sum of $s_{corr}$ , $\lambda_{i}$, $\lambda_{j}$ 
and \textit{Spiky} kernel over all neighbours. 
The parameters of $s_{corr}$; k, n and $ \nabla \mathbf{q}$, are defined as $k = 0.2, n = 4, |\nabla \mathbf{q}| = 0.2$. 
\\
At last, the total predicted position is updated and applied by adding the newly and previously estimated position, 
as seen on line 20 at Algorithm \ref{alg:overview}.

\subsubsection{Vorticity and viscosity}
The steps to calculate the vorticity and viscosity are fully described from Equation \ref{eq:Omega} to Equation \ref{eq:Viscosity}.
A local estimation of the vorticity is initially calculated and defined as $\omega$. The complete vorticity force is then determined 
by Equation \ref{eq:Viscosity}, where we use the force strength parameter as $\varepsilon = 1$. The force is then added to the total 
external force that is affecting the particles at the beginning of the time step, as seen on line 2 at Algorithm \ref{alg:overview}. 
\\
The new viscosity velocity is calculated by adding the old velocity with the sum of the smoothing kernel \textit{Poly6} multiplied by the velocity difference between 
the current particle and its neighbours. The viscosity strength parameter, as seen in Equation \ref{eq:Viscosity}, is $c = 0.0001$. 

\subsection{Rendering} To reduce frame times we use point sprites based upon
geometry shaders when rendering the particles. As we store the positions in
textures we use instance rendering of a single vertex and then use the instance
id to compute the texture coordinates used when querying the texture storing
the positions. The fetched position is then used as output of the vertex
shader. After, in the geometry shader, we spawn four new vertices making up the
surface plane of the particle. Finally in the fragment shader we compute the
normal in order to make the point sprite appear as a shaded sphere.

There is also a wireframe of a box rendered that contains the volume where
particle collisions can take place. We are limited due to the calculation of
Morton codes when constructing the grid.

The floor is rendered using a procedural checker pattern with anti aliasing and
is there to provide users with a cue for positional awareness.

% Something \cite{van2009screen}

\subsection{Parallelism}
In order to achieve real time performance the simulation had to be implemented
for parallel execution on GPUs (Graphics Processing Unit). This is made
possible through a technique called GPGPU (General-Purpose computing on
Graphics Processing Units). To maximize the throughput of GPUs fine-grained
parallelism is required. The use of particles together with PBD enables a
choice of the level of parallelism for a certain task. I.e. calculations can
either be done per particle, per constraint or per grid cell if a uniform grid
is used. We are mostly using the per particle or a particle-centric perspective
when performing the calculations, however we also use a constraint-centric
view, e.g. for solving collision constraints, and the per cell view when
constructing the grid.

