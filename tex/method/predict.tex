As seen in Algorithm ~\ref{alg:overview} the first thing that needs to be done at the beginning of a time step is to update the velocity of all particles by
considering their current velocity and any external forces applied to them as follows:

\begin{equation}
\label{eq:velocity}
\mathbf{v}_{i}^{new} = \mathbf{v}_{i} + (\mathbf{f}_{gravity} + \mathbf{f}_{i_{vorticity}})\Delta t
\end{equation}

We only consider two external forces $ f_{gravity} $ and $ f_{i_{vorticity}} $
that represents the force contributed by gravity and vorticity calculations
respectively. Note that vorticity is calculated at a later stage but is carried
over to the next time step so the $ f_{i_{vorticity}} $ that gets applied here
was calculated at the previous time step.

Once the velocity has been obtained it is used to predict a new particle
position according to the following expression:

\begin{equation}
\label{eq:predict}
\mathbf{x}_{i}^{*}= \mathbf{x}_{i} + \mathbf{v}_{i}^{new} \Delta t
\end{equation}

Having calculated the predicted position it is also necessary to verify that it
is actually a valid position. This is a requirement since the simulation is
limited to a certain volume. This step is performed by comparing the predicted
position against the boundary of the space containing the simulation. If a
particle is located outside the boundary its position is clamped to be inside
the volume.
