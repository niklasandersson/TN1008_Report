The work flow for solving the density constraint is closely tied with the theory starting at Equation \ref{eq:Ci}.
To determine $\lambda_{i}$, the components in Equation \ref{eq:LambdaEpsilon} are calculated separately in parallel
for each particle \textit{i}.
The numerator is calculated by following Equation \ref{eq:Ci}, which is a sum of the smoothing kernel Poly6 over
all neighbours. The smoothing kernel has a width $h_{Poly6} = 4.1$ and the rest density parameter, $\rho$, is set to be $\rho = 1200$.
The denominator is a sum of the norm of the Spiky kernel calculated over all neighbours
and is then set to the power of 2. The kernel width $h_{Spiky} = 6.1$ for Spiky works well and the
relaxation parameter $\varepsilon$ is set to $\varepsilon = 0.0001$.
\\
\newline
To extend the parallelization even more we rearrange and rewrite Equation \ref{eq:LambdaEpsilon} by including Equation \ref{eq:NablaC}. With the
new Equation \ref{eq:LambdaNew} we can observe that the difference between the sums is the norm of the gradient is applied
outside respectively inside of the sums:
\\
\begin{equation}
\label{eq:LambdaNew}
\lambda_i = \frac{- C_i(\mathbf{x}) }{ |\sum\limits_{k} \nabla \mathbf{x}_k W|^{2} + \sum\limits_{j} |-\nabla \mathbf{x}_j W|^2  + \varepsilon}.
\end{equation}
\\
\newline

We then continue to follow the algorithm overview at Algorithm \ref{alg:overview}, line 16, where the next step is to estimate the new position $\Delta \mathbf{x}^{*}_{i}$.
The predicted position is defined at Equation \ref{eq:DeltaPscorr}, containing a sum of $s_{corr}$ , $\lambda_{i}$, $\lambda_{j}$
and Spiky kernel over all neighbours.
The parameters of $s_{corr}$; k, n and $ \nabla \mathbf{q}$, are defined as $k = 0.2, n = 4, |\nabla \mathbf{q}| = 0.2$.

At last, the total predicted position is updated and applied by adding the newly and previously estimated position,
as seen on line 20 in Algorithm \ref{alg:overview}.

\subsubsection{Vorticity and viscosity}
The steps to calculate the vorticity and viscosity are fully described from Equation \ref{eq:Omega} to Equation \ref{eq:Viscosity}.
A local estimation of the vorticity is initially calculated and defined as $\omega$. The complete vorticity force is then determined
by Equation \ref{eq:Viscosity}, where we use the force strength parameter as $\varepsilon = 1$. The force is then added to the total
external force that is affecting the particles at the beginning of the time step, as seen on line 2 in Algorithm \ref{alg:overview}.

The new viscosity velocity is calculated by adding the old velocity with the sum of the smoothing kernel Poly6 multiplied by the velocity difference between
the current particle and its neighbours. The viscosity strength parameter, as seen in Equation \ref{eq:Viscosity}, is $c = 0.0001$.
